\chapter{Tóm tắt công thức}
\begin{minipage}[l]{0.45\textwidth}
	\section{Định luật Coulomb}
	$$F=\dfrac{k\left|q_1q_2\right|}{\varepsilon r^2}$$
	\section{Cường độ điện trường gây ra bởi điện tích $Q$}
	$$E=\dfrac{F}{\left|q\right|}=\dfrac{k\left|Q\right|}{\varepsilon r^2}$$
	\section{Nguyên lý chồng chất điện trường}
	$$\vec E=\vec{E}_1+\vec{E}_2+\dots+\vec{E}_n$$
	\section{Nguyên tắc tổng hợp vector cường độ điện trường}
	$$\vec E=\vec{E}_1+\vec{E}_2$$
	Độ lớn cường độ điện trường tổng hợp:
	$$E=\sqrt{E^2_1+E^2_2+2E_1E_2\cos\alpha}$$
	với $\alpha=\left(\vec{E}_1, \vec{E}_2\right)$.
	\begin{itemize}
		\item Nếu $\vec{E}_1\uparrow\uparrow\vec{E}_2$
		$$E=E_1+E_2.$$
		\item Nếu $\vec{E}_1\uparrow\downarrow\vec{E}_2$
		$$E=\left|E_1-E_2\right|.$$
		\item Nếu $\vec{E}_1\bot\vec{E}_2$
		$$E=\sqrt{E^2_1+E^2_2}.$$
		\item Nếu $E_1=E_2$ thì
		$$E=2E_1\cdot\cos\dfrac{\alpha}{2}.$$
	\end{itemize}
\end{minipage}
\begin{minipage}[c]{0.1\textwidth}
\
\end{minipage}
\begin{minipage}[l]{0.45\textwidth}
	\section{Thế năng điện}
	$$W_\text{A}=A_{\text{A}\infty}$$
	\section{Điện thế}
	$$V_\text{A}=\dfrac{W_\text{A}}{q}=\dfrac{A_{\text{A}\infty}}{q}$$
	\section{Hiệu điện thế}
	$$U_\text{AB}=V_\text{A}-V_\text{B}=\dfrac{A_\text{AB}}{q}$$
	\section{Liên hệ giữa cường độ điện trường và hiệu điện thế}
	$$E=\dfrac{U}{d}$$
	\section{Điện dung của tụ điện}
	$$C=\dfrac{Q}{U}$$
	\section{Điện dung của bộ tụ điện}
	\begin{itemize}
		\item \textbf{Bộ tụ ghép nối tiếp}
		\begin{eqnarray*}
			U&=&U_1+U_2+\dots+U_n\\
			Q&=&Q_1=Q_2=\dots=Q_n\\
			\dfrac{1}{C}&=&\dfrac{1}{C_1}+\dfrac{1}{C_2}+\dots+\dfrac{1}{C_n}
		\end{eqnarray*}
%	Nếu có $n$ tụ giống nhau thì $C_b=\dfrac{C}{n}$.
	\item \textbf{Bộ tụ ghép song song}
	\begin{eqnarray*}
		U&=&U_1=U_2=\dots=U_n\\
		Q&=&Q_1+Q_2+\dots+Q_n\\
		C&=&C_1+C_2+\dots+C_n
	\end{eqnarray*}
%Nếu có $n$ tụ giống nhau thì $C_b=nC$.
	\end{itemize}
\end{minipage}
\section{Năng lượng điện trường trong tụ điện}
$$W=\dfrac{1}{2}QU=\dfrac{1}{2}CU^2=\dfrac{Q^2}{2C}.$$
