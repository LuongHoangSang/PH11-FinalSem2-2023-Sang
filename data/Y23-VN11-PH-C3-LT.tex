\chapter{Tóm tắt lý thuyết}
\section{Sự nhiễm điện}
\subsection{Nhiễm điện do cọ xát}
Nhiễm điện do cọ xát là sự nhiễm điện khi các vật khác bản chất, trung hoà về điện được cọ sát với nhau. Khi đó hai vật sẽ nhiễm điện trái dấu.
\subsection{Nhiễm điện do tiếp xúc}
Nhiễm điện do tiếp xúc là sự nhiễm điện khi một vật trung hoà về điện đặt tiếp xúc với một vật nhiễm điện. Khi đó, hai vật sẽ nhiễm điện cùng dấu.
\subsection{Nhiễm điện do hưởng ứng}
Nhiễm điện do hưởng ứng là sự nhiễm điện khi một vật A (vật dẫn điện) trung hoà về điện đặt gần (không tiếp xúc) với một vật B nhiễm điện. Khi đó, đầu gần B của A tích điện trái dấu với vật B và đầu xa B của vật A tích điện cùng dấu với vật B. Khi đưa vật A ra xa vật B, vật A trở về trạng thái trung hoà như lúc đầu.
\section{Định luật Coulomb}
Lực tương tác giữa hai điện tích điểm đặt trong chân không có phương trùng với đường thẳng nối hai điện tích điểm đó, có độ lớn tỉ lệ thuận với tích độ lớn của các điện tích và tỉ lệ nghịch với bình phương khoảng cách giữa chúng.
$$F=\dfrac{k\left|q_1q_2\right|}{r^2}.$$
Trong đó:
\begin{itemize}
	\item $q_1, q_2$: giá trị đại số của hai điện tích, đơn vị trong hệ SI là coulomb $\left(\si{\coulomb}\right)$;
	\item $r$: khoảng cách giữa hai điện tích điểm, đơn vị trong hệ SI là mét $\left(\si{\meter}\right)$;
	\item $k$: hằng số phụ thuộc vào cách chọn hệ đơn vị của các đại lượng.\\
	Trong hệ SI, $k=\dfrac{1}{4\pi\varepsilon_0}=\SI{9E9}{\newton\meter^2/\coulomb^2}$ với $\varepsilon_0=\SI{8.861E-12}{\dfrac{\coulomb^2}{\newton\meter^2}}$ là hằng số điện.
\end{itemize}
Khi các điện tích đặt trong điện môi đồng tính thì lực tương tác giữa các điện tích điểm giảm đi $\varepsilon$ lần
$$F=\dfrac{k\left|q_1q_2\right|}{\varepsilon r^2}.$$
với $\varepsilon$ gọi là hằng số điện môi, chỉ phụ thuộc vào bản chất của môi trường.
\section{Điện trường}
\subsection{Khái niệm điện trường}
Điện trường là dạng vật chất bao quanh điện tích và truyền tương tác giữa các điện tích. Tính chất cơ bản của điện trường là tác dụng lực điện lên các điện tích khác đặt trong nó.
\subsection{Khái niệm cường độ điện trường}
Cường độ điện trường do điện tích $Q$ sinh ra tại một điểm M là đại lượng đặc trưng cho điện trường về mặt tác dụng lực tại điểm đó. Đây là một đại lượng vector và được xác định bởi biểu thức
$$\vec{E}=\dfrac{\vec{F}}{q}$$
Với $\vec{F}$ là lực tương tác tĩnh điện do điện tích $Q$ tác dụng lên một điện tích $q$ đặt tại điểm đó.\\
Như vậy, cường độ điện trường do điện tích $Q$ gây ra tại điểm M cách điện tích $Q$ một đoạn $r$ có
\begin{itemize}
	\item phương nằm trên đường thẳng nối điện tích và điểm M;
	\item chiều hướng ra xa điện tích nếu $Q>0$ và hướng lại gần điện tích nếu $Q<0$;
	\item độ lớn:
	$$E=\dfrac{k\left|Q\right|}{\varepsilon r^2}.$$
\end{itemize}
Trong hệ SI, cường độ điện trường có đơn vị là newton trên coulomb $\left(\si{\newton/\coulomb}\right)$ hoặc volt trên mét $\left(\si{\volt/\meter}\right)$.
\subsection{Đường sức điện}
Đường sức điện là đường mô tả điện trường sao cho tiếp tuyến tại mọi điểm bất kì trên đường sức cũng trùng với phương của vector cường độ điện trường tại điểm đó. Hướng của đường sức điện là hướng của điện trường tại điểm đó.\\
Đường sức điện có các đặc điểm sau:
\begin{itemize}
	\item Tại mỗi điểm trong điện trường có một và chỉ một đường sức điện đi qua;
	\item Các đường sức điện là các đường cong không kín. Nó xuất phát từ điện tích dương và tận cùng ở điện tích âm (hoặc ở vô cực);
	\item Nơi nào cường độ điện trường lớn hơn thì các đường sức điện ở đó được vẽ dày hơn, nơi nào cường độ điện trường nhỏ hơn thì các đường sức điện ở đó được vẽ thưa hơn.
\end{itemize}
\subsection{Điện trường đều}
Điện trường đều là điện trường mà cường độ điện trường tại mọi điểm có giá trị bằng nhau về độ lớn, giống nhau về phương và chiều.\\
Điện trường giữa hai bản phẳng, rộng, nhiễm điện trái dấu, đặt song song với nhau là điện trường đều. Cường độ điện trường giữa hai bản:
$$E=\dfrac{U}{d}.$$
Trong đó:
\begin{itemize}
	\item $U$: hiệu điện thế giữa hai bản phẳng, đơn vị trong hệ SI là volt $\left(\si{\volt}\right)$;
	\item $d$: khoảng cách giữa hai bản phẳng, đơn vị trong hệ SI là mét $\left(\si{\meter}\right)$.
\end{itemize}
\section{Điện thế và thế năng điện}
\subsection{Công của lực điện trong điện trường đều}
Công của lực điện tác dụng lên một điện tích không phụ thuộc vào dạng đường đi của điện tích mà chỉ phụ thuộc vào vị trí điểm đầu và điểm cuối của đường đi trong điện trường.\\
Công của lực điện tác dụng lên điện tích $q$ khi $q$ di chuyển từ A đến B trong điện trường đều:
$$A_\text{AB}=qE\overline{A'B'}$$
với $\overline{A'B'}=AB\cos\alpha$ là hình chiếu của AB lên phương đường sức điện trường và $\alpha=\left(\vec{E}, \overrightarrow{AB}\right).$
\subsection{Thế năng điện}
Thế năng điện của một điện tích $q$ tại một điểm trong điện trường đặc trưng cho khả năng sinh công của điện trường để dịch chuyển điện tích $q$ từ điểm đó ra xa vô cùng.
$$W_\text{A}=A_{\text{A}\infty}.$$
Trong hệ SI, thế năng điện có đơn vị là joule $\left(\si{\joule}\right).$
\subsection{Điện thế}
Điện thế tại một điểm trong điện trường là đại lượng đặc trưng cho thế năng điện tại vị trí đó và được xác định bằng công của lực điện dịch chuyển một đơn vị điện tích dương từ điểm đó ra vô cùng
$$V_\text{A}=\dfrac{W_\text{A}}{q}=\dfrac{A_{\text{A}\infty}}{q}.$$
\subsection{Hiệu điện thế}
Hiện điện thế giữa hai điểm A và B trong điện trường là đại lượng đặc trưng cho khả năng thực hiện công của điện trường để dịch chuyển một đơn vị điện tích giữa hai điểm đó và được xác định bởi biểu thức
$$U_\text{AB}=V_\text{A}-V_\text{B}=\dfrac{A_\text{AB}}{q}$$
Trong đó:
\begin{itemize}
	\item $U_\text{AB}$: hiệu điện thế giữa hai điểm A và B, đơn vị trong hệ SI là volt $\left(\si{\volt}\right)$;
	\item $A_\text{AB}$: công lực điện để dịch chuyển điện tích $q$ từ A đến B, đơn vị trong hệ SI là joule $\left(\si{\joule}\right)$;
	\item $q$: giá trị điện tích, đơn vị trong hệ SI là coulomb $\left(\si{\coulomb}\right)$.
\end{itemize}
\section{Tụ điện}
\subsection{Khái niệm tụ điện}
Tụ điện là hệ gồm hai vật dẫn đặt gần nhau và ngăn cách nhau bằng một lớp điện môi. Mỗi vật dẫn được gọi là một bản của tụ điện.
\subsection{Vai trò}
Trong mạch điện, tụ điện có nhiệm vụ tích điện và phóng điện.
\subsection{Điện dung}
Điện dung của tụ điện là đại lượng đặc trưng cho khả năng tích điện của tụ, kí hiệu là $C$ và được xác định bởi
$$C=\dfrac{Q}{U}$$
trong đó:
\begin{itemize}
	\item $C$: điện dung của tụ điện, đơn vị trong hệ SI là fara $\left(\si{\farad}\right)$;
	\item $Q$: độ lớn điện tích trên mỗi bản tụ, đơn vị trong hệ SI là coulomb $\left(\si{\coulomb}\right)$;
	\item $U$: hiệu điện thế giữa hai bản tụ, đơn vị trong hệ SI là volt $\left(\si{\volt}\right)$.
\end{itemize}
\subsection{Năng lượng}
Năng lượng được dự trữ trong tụ điện dưới dạng năng lượng điện trường
$$W=\dfrac{1}{2}QU=\dfrac{1}{2}CU^2=\dfrac{Q^2}{2C}$$
trong đó:
\begin{itemize}
	\item $W$: năng lượng điện trường trong tụ điện, đơn vị trong hệ SI là joule $\left(\si{\joule}\right)$;
	\item $Q$: điện tích mà tụ tích được, đơn vị trong hệ SI là coulomb $\left(\si{\coulomb}\right)$;
	\item $U$: hiệu điện thế giữa hai bản tụ, đơn vị trong hệ SI là volt $\left(\si{\volt}\right)$;
	\item $C$: điện dung của tụ điện, đơn vị trong hệ SI là fara $\left(\si{\farad}\right)$.
\end{itemize}